\documentclass{beamer}
\usepackage{amsfonts,amsmath}
\usetheme{sintef}
\usepackage{xeCJK}
\usepackage{graphicx}
\usepackage{booktabs}
\usepackage{array}
\usepackage{ragged2e}
\usepackage{hyperref}

% 自定义页码显示
\setbeamertemplate{footline}[frame number]
\usefonttheme[onlymath]{serif}

% 你模板里的背景(如不需要可注释)
\titlebackground*{assets/background_alternative}

% 彩色超链接(沿用你模板)
\newcommand{\hrefcol}[2]{\textcolor{cyan}{\href{#1}{#2}}}
\newcommand{\linkbtn}[2]{\textcolor{cyan}{\hyperlink{#1}{\underline{#2}}}}

% =========================
% Title info (edit as needed)
% =========================
\title{“Thinking with Images” 汇报1}
\author{尹超}
\date{\today}

\begin{document}
\maketitle
\addtocounter{framenumber}{-1}

% ============================================================
% 1) Paper 1: VisualToolBench
% ============================================================
\section{Paper 1: VisualToolBench}

\begin{frame}{Paper 1:VisualToolBench}
\begin{itemize}
  \item \alert{定位}:不是训练新模型,而是提出并评测 \alert{tool-enabled} 的“think-with-images”基准
  \item \alert{核心贡献}
  \begin{itemize}
    \item 任务设计覆盖单轮/多轮、多领域
    \item \alert{Rubric-based} 指标体系:可解释、可诊断(部分正确也能量化)
    \item 标准化工具集合(图像处理、搜索、计算等)用于统一评测
  \end{itemize}
\end{itemize}

% \begin{block}{对我们意义}
% 用它做 \alert{上线前压力测试}:验证 agent 的“会不会用工具 + 用得是否有效 + 最终是否可用”。
% \end{block}
\end{frame}

\begin{frame}{VisualToolBench:数据与任务结构}
\begin{columns}
\begin{column}{0.56\textwidth}
\begin{block}{数据规模}
\begin{itemize}
  \item 1,204 道开放式题(single-turn 603 / multi-turn 601)
\item 2,893 张图像
\item 7,777 条 rubrics
\item 5 大领域均衡:STEM/Medical/Finance/Sports/Generalist
\end{itemize}
\end{block}
\end{column}
\begin{column}{0.44\textwidth}
\begin{block}{任务结构}
\begin{itemize}
  \item Single-turn:
  \begin{itemize}
    \item Region Switch Q\&A
    \item Hybrid Tool-use
  \end{itemize}
  \item Multi-turn:
  \begin{itemize}
    \item Follow-up Test
    \item Temporal Reasoning
    \item Progressive Reasoning
  \end{itemize}
\end{itemize}
\end{block}
\end{column}
\end{columns}
\end{frame}

\begin{frame}{VisualToolBench:数据}
  \begin{figure}
    \centering
    \includegraphics[width=0.85\textwidth]{assets/1.png}
    \caption{VisualToolBench:各类任务示例(来源:论文 Figure 2)}
  \end{figure}
\end{frame}

\begin{frame}{VisualToolBench:评测指标(Rubric 体系)}
\begin{block}{Rubric 维度(5 类)}
Visual Understanding / Truthfulness / Instruction Following / Reasoning / Presentation
\end{block}

\begin{block}{计分方式(关键要点)}
\begin{itemize}
  \item 每题包含多条 rubric,含权重 $w \in \{1,2,3,4,5\}$
  \item Judge 判断每条 rubric 是否满足(0/1),按权重加权归一化得到任务分数
  \item 可派生 Pass/Fail(关键 rubric 决定):更接近“能不能用”的产品指标
\end{itemize}
\end{block}

\begin{block}{对我们:怎么用}
\begin{itemize}
  \item rubric score 用于 \alert{诊断}:错在视觉/真实性/遵循/推理/表达哪一类
  \item pass/fail 用于 \alert{可用性门槛}:上线前压测更友好
\end{itemize}
\end{block}
\end{frame}

% ---- Figure placeholder + bidirectional links
\begin{frame}{VisualToolBench:Rubric 评测示意}
\begin{figure}
  \centering
  \includegraphics[width=0.90\textwidth]{assets/2.png}
  \caption{评测流程}
\end{figure}
\vspace{4pt}
\linkbtn{vtbFigOneExp}{跳转:图表解读页}
\end{frame}



% \begin{frame}{VisualToolBench:开源与可跑性(baseline 价值判断)}
% \begin{block}{判断}
% \begin{itemize}
%   \item \alert{优势}:作为“终极靶场”非常强(多工具、多轮、rubric 可解释)
%   \item \alert{风险}:更像榜单/评测平台思路;是否支持离线一键跑需额外确认
% \end{itemize}
% \end{block}

% \begin{block}{建议用法(工程顺序)}
% 先用 VTBench 做日常回归,再用 DeepEyes 做 RL tool-use baseline,最后用 VTB 压测。
% \end{block}
% \end{frame}

% ============================================================
% 2) Paper 2: V-Thinker
% ============================================================
\section{Paper 2: V-Thinker}

\begin{frame}{Paper 2:V-Thinker}
\begin{itemize}
  \item \alert{定位}:训练一个通用的 \alert{interactive thinking with images} 模型
  \item \alert{两大贡献}
  \begin{itemize}
    \item Data Evolution Flywheel:合成并校验可执行的交互式数据(逐步改图/标注/推理)
    \item VTBench:分三层任务的评测集(感知 $\rightarrow$ 交互 $\rightarrow$ 推理)
  \end{itemize}
\end{itemize}

% \begin{block}{对我们意义}
% VTBench 的“分层诊断”非常适合做 \alert{内部回归基准}(定位问题在哪一层)。
% \end{block}
\end{frame}

\begin{frame}{V-Thinker:训练数据与 base model}
\begin{columns}
\begin{column}{0.52\textwidth}
\begin{block}{SFT 数据}
\begin{itemize}
  \item V-Perception-40K(感知对齐)
  \item V-Interaction-400K(交互对齐)
\end{itemize}
\end{block}

\begin{block}{RL 采样}
\begin{itemize}
  \item 从 We-Math2.0 / MMK12 / ThinkLite / V-Interaction-400K 采样 40K
\end{itemize}
\end{block}
\end{column}
\begin{column}{0.48\textwidth}
\begin{block}{Base / 对照}
\begin{itemize}
  \item 主要对照基线围绕 \alert{Qwen2.5-VL-7B} 展开(结果以 $\Delta$ 展示)
\end{itemize}
\end{block}

\begin{block}{开源}
\begin{itemize}
  \item GitHub:\hrefcol{https://github.com/We-Math/V-Thinker}{We-Math/V-Thinker}
\end{itemize}
\end{block}
\end{column}
\end{columns}
\end{frame}

\begin{frame}{The Data Evolution Flywheel framework}
\begin{figure}
  \centering
  \includegraphics[width=0.85\textwidth]{assets/3.png}
  \caption{Data Evolution Flywheel 框架}
\end{figure}
\end{frame}

\begin{frame}{Performance on VTBench and general reasoning}
\begin{figure}
  \centering
  \includegraphics[width=0.85\textwidth]{assets/4.png}
  \caption{Overall performance on VTBench and general reasoning}
\end{figure}
\end{frame}

\begin{frame}{VTBench:任务 $\rightarrow$ 输出 $\rightarrow$ 判分}
\begin{block}{VTBench 三层任务(每层 500,合计 1500 QA)}
\begin{itemize}
  \item \alert{Perception}:细粒度定位/坐标
  \item \alert{Instruction-Guided Interaction}:按指令画线/标注/改图
  \item \alert{Interactive Reasoning}:交互支撑的推理问答
\end{itemize}
\end{block}

\begin{block}{输出与判分口径}
\begin{itemize}
  \item Perception:输出 \alert{Python code} 画点/坐标;执行后图像 vs 标注图,\alert{LMM-as-judge}
  \item Interaction:输出 \alert{Python code} 执行交互;输出图 vs 标注图,\alert{LMM-as-judge}
  \item Reasoning:输出 \alert{final answer};\alert{LLM-as-judge} 判正确
\end{itemize}
\end{block}
\end{frame}





% \begin{frame}{V-Thinker:可跑性判断(baseline 视角)}
% \begin{block}{最值得先复用的部分}
% \begin{itemize}
%   \item \alert{VTBench 评测框架}(分层诊断 + code-output 判分)更容易快速落地
% \end{itemize}
% \end{block}

% \begin{block}{可能更重的部分}
% \begin{itemize}
%   \item Data Evolution Flywheel 的完整复现依赖生成器/校验器/工具库细节(工程投入更大)
% \end{itemize}
% \end{block}
% \end{frame}

% ============================================================
% 3) Paper 3: DeepEyes
% ============================================================
\section{Paper 3: DeepEyes}

\begin{frame}{Paper 3:DeepEyes}
\begin{itemize}
  \item \alert{定位}:用强化学习直接激励 “thinking with images”
  \item \alert{核心思路}:用最小工具闭环(zoom-in)把“看不清”变成可学习动作
  \item \alert{训练算法}:GRPO(group relative policy optimization)
\end{itemize}
\begin{block}{意义}

主动感知 (Active Perception): 模型自己决定“看哪里”,而不是被动接受固定输入。

多步规划 (Multi-step Planning): 将复杂问题拆解为定位、放大、对比、判断四个阶段。

工具使用 (Tool Use): 将“放大/裁剪”作为一种外部工具来增强自身的感知上限。
\end{block}
\end{frame}

\begin{frame}{DeepEyes:方法概览}
\begin{figure}
  \centering
  \includegraphics[width=0.75\textwidth]{assets/5.png}
  \caption{DeepEyes 方法概览}
\end{figure}
\end{frame}

\begin{frame}{DeepEyes:训练数据与 Base model}
\begin{columns}
\begin{column}{0.55\textwidth}
\begin{block}{训练数据三块}
\begin{itemize}
  \item Fine-grained:V* training set(小目标/细字/局部细节)
  \item Chart:ArxivQA(图表/曲线等结构化视觉)
  \item Reason:ThinkLite-VL(推理广度)
\end{itemize}
\end{block}
\end{column}
\begin{column}{0.45\textwidth}
\begin{block}{Base / 对照}
\begin{itemize}
  \item 明确围绕 \alert{Qwen2.5-VL-7B} 做数据选择与对照
\end{itemize}
\end{block}

\begin{block}{开源}
\begin{itemize}
  \item GitHub:\hrefcol{https://github.com/Visual-Agent/DeepEyes}{Visual-Agent/DeepEyes}
\end{itemize}
\end{block}
\end{column}
\end{columns}
\end{frame}

\begin{frame}{DeepEyes:Reward}
\begin{block}{三项 reward 组成(工程要点)}
\begin{itemize}
  \item Accuracy reward:最终答案正确性
  \item Formatting reward:输出格式可解析(防止 policy 崩坏)
  \item \alert{Conditional tool bonus}:仅当答案正确且至少使用一次工具才给 bonus
\end{itemize}
\end{block}

\[
R(\tau) = R_{\text{acc}}(\tau) + R_{\text{format}}(\tau) + \mathbb{I}_{R_{\text{acc}}(\tau)>0} \cdot R_{\text{tool}}(\tau)
\]

\begin{block}{为什么重要}
\begin{itemize}
  \item 防止“乱用工具”刷分;鼓励“\alert{正确且有效} 的工具调用”
  \item 特别适合做:\alert{工具策略是否真实提升成功率} 的验证闭环
\end{itemize}
\end{block}
\end{frame}

\begin{frame}{DeepEyes:评测覆盖面(用于综合回归)}
\begin{block}{高分辨/细粒度}
V*Bench、HR-Bench(4K/8K)——验证 zoom-in 是否提升细节感知
\end{block}

\begin{figure}
  \centering
  \includegraphics[width=0.85\textwidth]{assets/6.png}
  \caption{高分辨率基准测试结果}
\end{figure}
\end{frame}

\begin{frame}{DeepEyes:评测覆盖面(用于综合回归)}

\begin{block}{Grounding / Hallucination}
refCOCO/refCOCO+/refCOCOg、ReasonSeg、POPE ——验证定位与幻觉抑制
\end{block}

\begin{figure}
  \centering
  \includegraphics[width=0.85\textwidth]{assets/7.png}
  \caption{在定位(Grounding)与幻觉(Hallucination)基准测试上的结果}
\end{figure}

\end{frame}
\begin{frame}{DeepEyes:评测覆盖面(用于综合回归)}
\begin{block}{推理任务}
MathVista/MathVision/MathVerse/We-Math 等 ——验证泛化到推理 QA
\end{block}

\begin{figure}
  \centering
  \includegraphics[width=0.85\textwidth]{assets/8.png}
  \caption{多模态推理基准测试结果}
\end{figure}
\end{frame}





% ============================================================
% 4) Paper 4: SAM-R1
% ============================================================
\section{Paper 4: SAM-R1}

\begin{frame}{Paper 4:SAM-R1}
\begin{itemize}
  \item \alert{定位}:多模态分割 / 指代分割(referring segmentation)的 RL 对齐
  \item \alert{核心范式}:MLLM 先输出 bbox/point 等中间表示,再交给 \alert{SAM2-Large} 生成 mask
  \item \alert{关键贡献}:用 SAM 的 mask 质量构造 reward,把分割质量显式反馈给 MLLM
\end{itemize}

\begin{figure}
  \centering
  \includegraphics[width=0.75\textwidth]{assets/9.png}
  \caption{SAM-R1 方法概览}
\end{figure}

\end{frame}

\begin{frame}{SAM-R1:方法概览}
\begin{figure}
  \centering
  \includegraphics[width=0.85\textwidth]{assets/10.png}
\end{figure}
\end{frame}

\begin{frame}{SAM-R1:训练数据 / Base model / 输出格式}
\begin{columns}
\begin{column}{0.55\textwidth}
\begin{block}{训练数据}
\begin{itemize}
  \item RefCOCOg train 随机采样 3,000
  \item In-domain:RefCOCOg test
  \item OOD:RefCOCO testA、RefCOCO+ testA
  \item 额外:ReasonSeg(zero-shot)
\end{itemize}
\end{block}
\end{column}
\begin{column}{0.45\textwidth}
\begin{block}{Base}
\begin{itemize}
  \item MLLM:\alert{Qwen2.5-VL-7B}
  \item Segmenter:\alert{SAM2-Large}
\end{itemize}
\end{block}

\begin{block}{输出格式(必须对齐)}
\begin{itemize}
  \item bbox + reference point + textual flag(结构化)
  \item 再由 SAM 产 mask
\end{itemize}
\end{block}
\end{column}
\end{columns}
\end{frame}

\begin{frame}{SAM-R1:指标}
\begin{block}{评测指标}
\begin{itemize}
  \item \alert{gIoU}:per-image IoU 平均
  \item \alert{cIoU}:累计 intersection / 累计 union
\end{itemize}
\end{block}

\begin{figure}
  \centering
  \includegraphics[width=0.80\textwidth]{assets/11.png}
  \caption{在 RefCOCOg/RefCOCO/RefCOCO+ 上的评测结果}
\end{figure}
\end{frame}


\begin{frame}{SAM-R1:指标}

\begin{block}{训练 reward(关键)}
\begin{itemize}
  \item 将(bbox+point)交给 SAM 得到 mask,与 GT mask 算 IoU
  \item 按 IoU 分段给 reward(例如 $>0.8, 0.7\sim0.8, 0.5\sim0.7, \text{else}$)
\end{itemize}
\end{block}

\begin{block}{为什么重要}
把“分割质量”变成可学习信号,避免只在语言空间里对齐。
\end{block}
\end{frame}



% ============================================================
% 5) Cross-paper comparison
% ============================================================
\section{Cross-paper Comparison}

% \begin{frame}{横向对比:四篇论文的“任务范式”定位}
% \begin{columns}
% \begin{column}{0.5\textwidth}
% \begin{block}{评测靶场(压测)}
% \begin{itemize}
%   \item VisualToolBench:多工具、多轮、rubric 诊断强
% \end{itemize}
% \end{block}

% \begin{block}{交互式推理(code 可执行)}
% \begin{itemize}
%   \item V-Thinker + VTBench:Perception/Interaction/Reasoning 分层评测
% \end{itemize}
% \end{block}
% \end{column}
% \begin{column}{0.5\textwidth}
% \begin{block}{最小 RL baseline(单工具)}
% \begin{itemize}
%   \item DeepEyes:zoom-in + GRPO + conditional tool bonus
% \end{itemize}
% \end{block}

% \begin{block}{分割 RL(IoU 可度量)}
% \begin{itemize}
%   \item SAM-R1:bbox/point $\rightarrow$ SAM mask,gIoU/cIoU + OOD
% \end{itemize}
% \end{block}
% \end{column}
% \end{columns}
% \end{frame}

\begin{frame}{对齐表:数据集/Benchmark $\rightarrow$ 任务 $\rightarrow$ 输出 $\rightarrow$ 判分}
\scriptsize % 改为 tiny 以确保不溢出
\setlength{\tabcolsep}{1.5pt} % 减小列间距
\renewcommand{\arraystretch}{1.2} % 稍微增加行高
\begin{table}
\centering
\begin{tabular}{>{\RaggedRight\arraybackslash}p{0.12\textwidth}
                >{\RaggedRight\arraybackslash}p{0.20\textwidth}
                >{\RaggedRight\arraybackslash}p{0.36\textwidth}
                >{\RaggedRight\arraybackslash}p{0.28\textwidth}}
\toprule
\textbf{论文} & \textbf{数据集/Benchmark} & \textbf{任务类型 \& 输出格式} & \textbf{判分方式/指标} \\
\midrule
VisualToolBench & VISUALTOOLBENCH &
Tool-enabled 单/多轮;输出:最终自然语言 answer(可含工具轨迹) &
Rubric(0/1)+ weight 加权;LLM-as-judge 逐条 rubric 判定 \\
\midrule
V-Thinker & VTBench(1500 QA,3 类任务) &
Perception:输出 Python code(画点/坐标);
Interaction:输出 Python code(执行交互);
Reasoning:输出 final answer &
Perception/Interaction:执行图 vs 标注图,LMM-as-judge;
Reasoning:LLM-as-judge \\
\midrule
DeepEyes & V* / HR-Bench / refCOCO* / POPE / ReasonSeg / Math* &
交互式推理(zoom-in);输出:\texttt{bbox\_2d} 工具调用(可多次)+ 最终 answer &
多按各 benchmark 官方协议;覆盖高分辨、grounding、hallucination 等 \\
\midrule
SAM-R1 & RefCOCOg / RefCOCO / RefCOCO+ / ReasonSeg &
指代分割;输出:bbox + point + flag(结构化)并交给 SAM 产 mask &
评测:gIoU/cIoU;
训练 reward:IoU 分段(SAM mask vs GT) \\
\bottomrule
\end{tabular}
\end{table}
\end{frame}

% \begin{frame}{Baseline 建议:最小可行路线(不堆叠,能落地)}
% \begin{block}{Step 1:统一接口(Qwen2.5-VL-7B + 工具)}
% \begin{itemize}
%   \item 工具先从 \alert{zoom-in/crop} 做起(最小闭环)
%   \item 记录每步工具输入/输出/耗时(为 rubric/错误归因做准备)
% \end{itemize}
% \end{block}

% \begin{block}{Step 2:分层评测回归(VTBench)}
% \begin{itemize}
%   \item 先定位是 Perception、Interaction 还是 Reasoning 层的问题
% \end{itemize}
% \end{block}

% \begin{block}{Step 3:训练对齐(DeepEyes 风格 RL)}
% \begin{itemize}
%   \item 用 conditional tool bonus 确保工具使用“有效且正确”
% \end{itemize}
% \end{block}

% \begin{block}{Step 4:上线前压力测试(VisualToolBench)}
% \begin{itemize}
%   \item 用 rubric score + pass/fail 做最终可用性门槛
% \end{itemize}
% \end{block}
% \end{frame}

% \begin{frame}{Q\&A}
% \centering
% {\Large Q\&A}\\
% \vspace{10pt}
% \small(需要我也可以再给一页“风险清单”:judge 偏差、tool misuse、offline 可跑性等)
% \end{frame}

\end{document}
